\documentclass[10pt, french]{article}
\usepackage[landscape, hmargin=1cm, vmargin=1.7cm]{geometry}
\usepackage{multicol}


\usepackage{tikz}

%% -----------------------------
%% tcolorbox configuration
%% -----------------------------
\usepackage[most]{tcolorbox}
\tcbuselibrary{xparse}
\tcbuselibrary{breakable}

%%
%% Coloured box "definition" for definitions
%%
\DeclareTColorBox{definition}{ o }				% #1 parameter
{
	colframe=blue!60!green,colback=blue!5!white, % color of the box
	breakable, 
	pad at break* = 0mm, 						% to split the box
	title = {#1},
	after title = {\large \hfill \faBook}
}
%%
%% Coloured box "algo" for algorithms
%%
\newtcolorbox{algo}[ 1 ]
{
	colback = blue!5!white,
	colframe = blue!75!black,
	fonttitle = \bfseries,
	title = {#1},
	valign = center
}
\newtcolorbox{algo2}[ 1 ]
{
	colback = blue!5!white,
	colframe = teal!75!black,
	fonttitle = \bfseries,
	title = {#1},
	valign = center
}
\newtcolorbox{algo3}[ 1 ]
{
	colback = blue!5!white,
	colframe = purple!60!black,
	fonttitle = \bfseries,
	title = {#1},
	valign = center
}
%%
%% Coloured box "formula" for formulas
%%
\newtcolorbox{formula}[ 1 ]
{
	colback = green!5!white,
	colframe = green!75!black,
	fonttitle = \bfseries,title=#1
}


\begin{document}

\section*{Cheminement}

\begin{multicols*}{2}

\begin{algo2}{Complétés automne 2017}
\begin{enumerate}
	\item[] \textbf{CTB-1000}: Comptabilité générale
	\item[] \textbf{ACT-1899}: Équivalence de crédits \textit{(cours à option de règle 2)}
	\item[] \textbf{ACT-1899}: Équivalence de crédits \textit{(cours à option de règle 2)}
	\item[] \textit{COM-1899}: Équivalence de crédits \textit{(Cours d’anglais ou de langues)}
\end{enumerate}
\end{algo2}

%\begin{tcbraster}[raster columns=2, rows = 2, raster equal height]

\begin{algo}{Complétés automne 2018}
\begin{enumerate}
	\item[] \textbf{ACT-1000}: Introduction à l’actuariat I
	\item[] \textbf{ACT-1001}: Mathématiques financières
	\item[] \textbf{ACT-1002}: Analyse probabiliste des risques actuariels
	\item[] \textbf{ACT-1003}: Compléments de mathématiques
	\item[] \textbf{IFT-1902}: Prog. avec R pour l'analyse de données
\end{enumerate}
\end{algo}



\begin{algo}{Complétés hiver 2019}
\begin{enumerate}
	\item[] \textbf{ACT-1006}: Gestion du risque financier I              
	\item[] \textbf{ACT-2000}: Analyse statistique des risques actuariels 
	\item[] \textbf{ACT-2001}: Introduction à l’actuariat II
	\item[] \textbf{ACT-2002}: Méthodes numériques                        
\end{enumerate}
\end{algo}



\begin{algo2}{Habituellement complétés hiver 2019}
\begin{enumerate}
	\item[] \textbf{CTB-1000}: Comptabilité générale
\end{enumerate}
\end{algo2}

\begin{algo2}{Complétés été 2019}
\begin{enumerate}
	\item[] \textbf{ECN-1000}: Principes de microéconomie
	\item[] \textbf{ECN-1010}: Principes de macroéconomie
\end{enumerate}
\end{algo2}



\begin{algo}{Complétés automne 2019}
\begin{enumerate}
	\item[] \textbf{ACT-2003}: Modèles linéaires                                 
	\item[] \textbf{ACT-2004}: Mathématiques actuarielles vie I       
	\item[] \textbf{ACT-2005}: Mathématiques actuarielles IARD I     
	\item[] \textbf{ACT-2009}: Processus stochastiques                           
\end{enumerate}
\end{algo}

\begin{algo2}{Habituellement complétés automne 2019}
\begin{enumerate}
	\item[] \textbf{ECN-1000}: Principes de microéconomie            
\end{enumerate}
\end{algo2}

\begin{algo3}{À faire hiver 2020}
\begin{enumerate}
	\item[] \textbf{ACT-2007}: Mathématiques actuarielles vie II         
	\item[] \textbf{ACT-2008}: Mathématiques actuarielles IARD II        
	\item[] \textbf{ACT-2011}: Gestion du risque financier II                     
	\item[] \textcolor{green!60!black}{\textbf{ACT-1005}: Analyse et traitement collectif du risque} 
	\item[] \textcolor{green!60!black}{\textbf{FRN-2900}: Communication en actuariat}
\end{enumerate}
\end{algo3}

\begin{formula}{À faire été 2020}
\begin{enumerate}
	\item[] \textbf{FRN-3003}: Français avancé: grammaire et rédaction II
	\item[] \textit{\textcolor{brown!60!black}{\textbf{ACT-XYXY}: cours à option de règle 2 \textit{(3 de 3)}}}
\end{enumerate}
\end{formula}

\newpage

\begin{algo3}{À faire automne 2020}
\begin{enumerate}
	\item[] \textbf{ACT-3000}: Théorie du risque
	\item[] \textbf{ACT-XYXY}: cours à option de règle 1 \textit{(1 de 3)}
	\item[] \textbf{ACT-XYXY}: cours à option de règle 1 \textit{(2 de 3) }
\end{enumerate}
\end{algo3}

\begin{algo2}{Habituellement complétés automne 2020}
\begin{enumerate}
	\item[] \textbf{ACT-1899}: cours à option de règle 2                \textit{(1 de 3)}
	\item[] \textit{ACT-42}: Cours d’anglais ou de langues
\end{enumerate}
\end{algo2}

\begin{algo3}{À faire hiver 2021}
\begin{enumerate}
	\item[] \textbf{ACT-3001}: Législation et responsabilité prof. en act.
	\item[] \textbf{ACT-4105/ACT-3114/ACT-2101}: cours à option de règle 1 \textit{(3 de 3)}
	\item[] \textcolor{brown!60!black}{\textbf{ACT-XYXY}: cours à option de règle 2 \textit{(3 de 3)}}
	\item[] \textit{\textcolor{green!60!black}{\textbf{ACT-1005}: Analyse et traitement collectif du risque}}
	\item[] \textit{\textcolor{green!60!black}{\textbf{FRN-2900}: Communication en actuariat}}
\end{enumerate}
\end{algo3}

\begin{algo2}{Habituellement complété hiver 2021}
\begin{enumerate}
	\item[] \textbf{ACT-1899}: cours à option de règle 2 \textit{(2 de 3)}
	\item[] \textbf{ECN-1010}: Principes de macroéconomie
\end{enumerate}
\end{algo2}
 
\end{multicols*}

\section*{Cheminement considéré}

\begin{multicols*}{2}

\end{multicols*}

\end{document}
